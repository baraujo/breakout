\documentclass{beamer}

%\usepackage[utf8]{inputenc}
\usepackage[T1]{fontenc}
\usepackage{fontspec}
\usepackage[brazil]{babel}
\usepackage{listings}
\usepackage{graphicx}
\usepackage{color}
\definecolor{editorGray}{rgb}{0.95, 0.95, 0.95}
\definecolor{editorOcher}{rgb}{1, 0.5, 0} % #FF7F00 -> rgb(239, 169, 0)
\definecolor{editorGreen}{rgb}{0, 0.5, 0} % #007C00 -> rgb(0, 124, 0)
\usetheme[outer/progressbar=foot]{Metropolis}
\setsansfont[BoldFont={Source Sans Pro Semibold},Numbers={OldStyle}]{Source Sans Pro}

%\makeatletter
%\def\verbatim@font{\ttfamily}
%\makeatother

\lstset{%
    % Basic design
    backgroundcolor=\color{editorGray},
    basicstyle={\scriptsize\ttfamily},   
	%basicstyle={\scriptsize},   
    frame=l,
    % Line numbers
    xleftmargin={0.75cm},
   % numbers=left,
    stepnumber=1,
    firstnumber=1,
    numberfirstline=true,
    % Code design   
    keywordstyle=\color{blue}\bfseries,
    commentstyle=\color{darkgray}\ttfamily,
    ndkeywordstyle=\color{editorGreen}\bfseries,
    stringstyle=\color{editorOcher},
    % Code
    language=csh,
    alsodigit={.:;},
    tabsize=2,
    showtabs=false,
    showspaces=false,
    showstringspaces=false,
    extendedchars=true,
    breaklines=true,
    morekeywords={
    	abstract, event, new, struct,
    	as, explicit, null, switch,
    	base, extern, object, this,
    	bool, false, operator, throw,
    	break, finally, out, true,
    	byte, fixed, override, try,
    	case, float, params, typeof,
    	catch, for, private, uint,
    	char, foreach, protected, ulong,
    	checked, goto, public, unchecked,
    	class, if, readonly, unsafe,
    	const, implicit, ref, ushort,
    	continue, in, return, using,
    	decimal, int, sbyte, virtual,
    	default, interface, sealed, volatile,
    	delegate, internal, short, void,
    	do, sizeof, while,
    	double, lock, stackalloc,
    	else, long, static,
    	enum, namespace, string
    },
    literate={ç}{{\c{c}}}1 {é}{{\'e}}1 {á}{{\'a}}1 {ã}{{\~a}}1 {ó}{{\'o}}1 {í}{{\'i}}1
}

\begin{document}
\title{Programação Básica em C\# para Unity}   
\author{Bruno dos Santos de Araújo, MSc} 
\date{\today} 

%\AtBeginSection[]
%{
%  \begin{frame}
%  \frametitle{Sumário}
%  \tableofcontents[currentsection]
%  \end{frame}
%}

\frame{\titlepage} 

\frame{\frametitle{Sumário}\tableofcontents} 

\section{Introdução}

\begin{frame}[fragile]{Introdução}
	\begin{itemize}
		\item Este treinamento tem os seguintes objetivos:
		\begin{itemize}
			\item Entender o conceito de \verb|game loop| e como se aplica ao Unity;
			\item Mostrar a estrutura básica do tipo de script mais comum no Unity: \verb|MonoBehaviour|;
			\item Implementar uma versão simples porém funcional do jogo Breakout (também conhecido como Arkanoid).
		\end{itemize}
		\item Existem diversas abordagens possíveis para os problemas que iremos resolver, neste treinamento utilizaremos as abordagens mais simples e didáticas e não necessariamente as mais eficientes.
	\end{itemize}
\end{frame}

\section{Game loop}

\begin{frame}[fragile]{Game loop}
	\begin{itemize}
		\item Quase todos os jogos utilizam um algoritmo chamado \verb|game loop|, que basicamente possui 3 funções:
		\begin{itemize}
			\item Ler e processar a entrada do usuário
			\item Atualizar o estado interno do jogo
			\item Desenhar a tela
		\end{itemize}
		\item O número de vezes que esse algoritmo é executado por segundo determina a taxa de \textbf{quadros por segundo} do jogo, ou \textbf{FPS} (frames per second)
		\begin{itemize}
			\item 30 FPS: a cada ~33ms
			\item 60 FPS: a cada ~16ms
		\end{itemize}
	\end{itemize}
\end{frame}

\begin{frame}[fragile]{Game loop}
	\begin{itemize}
		\item Visualização do algoritmo:
	\end{itemize}		
	\begin{lstlisting}
while(gameIsRunnning) {
	readInput();
	update();
	render();
}
\end{lstlisting}
	\begin{itemize}
		\item \verb|readInput()|: lê a entrada do usuário
		\item \verb|update()|: atualiza o estado interno do jogo
		\item \verb|render()|: desenha a tela
	\end{itemize}
\end{frame}

\begin{frame}[fragile]{Game loop}
%	\begin{itemize}
%		\item Visualização do algoritmo:
%	\end{itemize}		
%	
%	\begin{lstlisting}
%while(gameIsRunnning) {
%	readInput();
%	update();
%	render();
%}
%	\end{lstlisting}
	\begin{itemize}
		\item Ao implementar código de gameplay no Unity, precisamos sempre implementar os equivalentes a \verb|readInput()| e \verb|update()| dentro de uma subclasse de \verb|MonoBehaviour|.
		\begin{itemize}
			\item \verb|render()| é implementada via shaders; mesmo que você não escreva nenhum, o Unity utiliza shaders padrão
		\end{itemize}
	\begin{itemize}
		\item Boa referência sobre o assunto: \url{https://gameprogrammingpatterns.com/game-loop.html}
	\end{itemize}
	\end{itemize}
\end{frame}

\section{Estrutura de um MonoBehaviour}

\begin{frame}[fragile]{Estrutura de um MonoBehaviour}
	\begin{itemize}
		\item \verb|MonoBehaviour| é a classe da qual se deriva a maioria dos componentes dentro do Unity.
		\item Ao criarmos um novo \verb|MonoBehaviour|, seja diretamente como componente num GameObject ou na hierarquia, nos deparamos com o seguinte código:		
	\end{itemize}
	\begin{lstlisting}
public class NewBehaviourScript : MonoBehaviour
{
	void Start()
	{
	}
	
	void Update()
	{	
	}
}	
	\end{lstlisting}
	\begin{itemize}
		\item Podemos observar duas funções que foram criadas por padrão: \verb|Start()| e \verb|Update()|.
	\end{itemize}
\end{frame}

\begin{frame}[fragile]{Estrutura de um MonoBehaviour}
	\begin{itemize}
		\item \verb|Start()|: esta função é chamada assim que o GameObject é ativado em uma cena;
		\begin{itemize}
			\item Normalmente utilizada para inicializar quaisquer variáveis e chamar funções de inicialização.
		\end{itemize}
		\item \verb|Update()|: esta função é chamada a cada quadro, e é utilizada para atualizar o estado interno do componente. 
		\begin{itemize}
			\item No caso de ser um objeto controlável pelo jogador, também é onde se leem as entradas (teclas, botões, etc);
			\item O tempo entre chamadas é aproximadamente o mesmo e pode variar de acordo com a quantidade de objetos na cena.
		\end{itemize}
	\end{itemize}
\end{frame}

\begin{frame}[fragile]{Estrutura de um MonoBehaviour}
	\begin{itemize}
		\item Outras funções comumente utilizadas:
		\begin{itemize}
			\item Awake(): é chamada no ato da criação do GameObject, sempre antes de Start(), independentemente do GameObject estar ativado ou não;
			\item FixedUpdate(): semelhante a Update() mas é chamada em tempos fixos a cada passo da Física, que por padrão é 20ms;
			\begin{itemize}
				\item Utilizada principalmente quando se utilizam funções que agem na física do GameObject.
			\end{itemize}
			\item OnEnable() e OnDisable(): chamadas na ativação e desativação do GameObject, respectivamente;
		\end{itemize}
	\end{itemize}
\end{frame}

\end{document}

